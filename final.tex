\documentclass[11pt]{article}

\usepackage[margin=1in]{geometry} 
\usepackage{amsmath,amsthm,amssymb,amsfonts}

\usepackage[utf8]{inputenc}
\usepackage[T1]{fontenc}
\usepackage[spanish]{babel}

\usepackage[bitstream-charter]{mathdesign}

\usepackage{mathrsfs}
\usepackage{enumitem}
\usepackage{microtype}
\usepackage{tocbibind}
\usepackage[
    %textcolor=red,
    linecolor=color,
    bordercolor=color,
    backgroundcolor=white,
]{todonotes}
\usepackage{titlesec}
\titleformat{\section}[block]{\Large\bfseries}{}{0em}{}
\titleformat{\subsection}[block]{\large\bfseries\scshape\filcenter}{}{1em}{}

\usepackage{thmtools,xcolor}
\usepackage{fancyhdr}
\pagestyle{fancy}
\usepackage[misc]{ifsym}
\usepackage{tcolorbox}
\tcbuselibrary{theorems}

\usepackage{graphicx}
\usepackage{lipsum}

\usepackage{tikz}
\usepackage{tikz-cd}
\usetikzlibrary{arrows}
\usetikzlibrary{matrix}

\definecolor{color}{RGB}{189, 77, 77}
\renewcommand\qedsymbol{$\paint{\blacktriangleleft}$}
\declaretheoremstyle[
  spaceabove = 6pt,
  spacebelow = 6pt,
  headfont=\color{color}\normalfont\bfseries,
  notefont=\color{color}\normalfont\bfseries
]{colored}
\theoremstyle{colored}
\DeclareMathOperator{\sop}{sop}
\newtheorem{definition}{\scshape Definición}
\newtheorem{theorem}{\scshape Teorema}
\newtheorem*{theorem*}{Teorema}
\newtheorem{proposition}{\scshape Proposición}
\newtheorem{corollary}{\scshape Corolario}
\newtheorem{lemma}{\scshape Lema}
\newtheorem{remark}{\scshape Observación}
\newtheorem{example}{\scshape Ejemplo}
\newtheorem{exercise}{Ejercicio}
\newtheorem{exam-exercise}{Ejercicio}

\usepackage{hyperref}
\hypersetup{
    colorlinks,
    citecolor=color,
    filecolor=color,
    linkcolor=color,
    urlcolor=color,
}

\newcommand{\N}{\mathbb{N}}
\newcommand{\Z}{\mathbb{Z}}
\newcommand{\Q}{\mathbb{Q}}
\newcommand{\R}{\mathbb{R}}
\newcommand{\C}{\mathscr{C}}
\newcommand{\D}{\mathbb{D}}
\newcommand{\Ss}{\mathbb{S}}
\renewcommand{\k}{\Bbbk}
\newcommand{\M}[2]{\mathsf{M}_{#1}#2}
\newcommand{\im}{\operatorname{im}}
\newcommand{\id}{\operatorname{id}}
\newcommand{\eps}{\varepsilon}
\newcommand{\nat}[1]{[\![#1]\!]}
\newcommand{\natzero}[1]{\nat{#1}_0}
\newcommand{\ol}{\overline}
\newcommand{\tint}[1]{\stackrel{o}{#1}}
\newcommand{\cat}[1]{\mathsf{#1}}
\newcommand{\test}{\mathscr{D}}
\newcommand{\dist}{\test'}
\newcommand{\ev}[1]{\langle #1 \rangle}
\newcommand{\ip}[1]{( #1 )}
\newcommand{\guill}[1]{«#1»}
\renewcommand{\L}{\mathscr{L}}
\newcommand{\paint}[1]{\color{color}{#1}}
\newcommand{\tpaint}[1]{\paint{\textbf{#1}}}
\newcommand{\paintline}{\begin{center}
$\paint{
\rule{400pt}{0.5pt}
}$
\vspace{10pt}
\end{center}}
\tcbset{colback=color!25,colframe=black!0, sharp corners = all}

%-----------------------
\lhead{Ecuaciones Diferenciales}
\rhead{Examen Final}
\begin{document}

\author{Guido Arnone}
\title{$\tpaint{\scshape Ecuaciones Diferenciales - Examen Final}$\\
\Large El teorema de Malgrange-Ehrenpreis}
\date{27 de diciembre de 2019}

\maketitle

Recordemos que dado un abierto $U \subset \R^n$ y una distribución $T \in \dist(U)$, se define su \textit{derivada débil} con respecto a cierta coordenada $i \in \nat{n}$ por
\[
\ev{\partial_iT,\varphi} := -\ev{T,\partial_i\varphi}. 
\]

Usando partes, esto coincide con la noción usual para funciones suaves: la derivada débil de una función suave coincide con la distribución que induce su derivada usual. Dado un polinomio $p = \sum_{\alpha}c_\alpha x^\alpha \in \R[X_1, \dots, X_n]$, se puede considerar el operador \textit{operador diferencial parcial lineal con coeficientes constantes} asociado a $p$,
\begin{align*}
\L \colon & \dist \longrightarrow \dist\\
& T \mapsto \sum_{\alpha}c_\alpha D^\alpha T.
\end{align*}
Se define también el \textit{operador adjunto formal} de $\L$ como
\begin{align*}
\L^\ast \colon & \dist \longrightarrow \dist\\
& T \mapsto \sum_{\alpha}(-1)^{|\alpha|}c_\alpha D^\alpha T,
\end{align*}
el cual satisface $\ev{\L T, \varphi} = \ev{T,\L^\ast \varphi}$ para todo $T \in \dist$ y $\varphi \in \test$. Análogamente se puede considerar la (co)restricción de $\L$ y $\L^\ast$ a  $\mathscr{C}_c^\infty(U)$.

Una \textit{solución fundamental} para $\L$ es una distribución $T \in \dist$ que satisface la ecuación
\[
\L T = \delta_0
\]
con $\delta_0$ la delta de Dirac. Esto es reminiscente al la ecuación de Laplace,
\[
\Delta u = 0
\]
para la cual existe una solución $\Phi \in \C^2(\R^n \setminus \{0\})$ definida por
\[
\Phi(x) = \begin{cases}
-\frac{1}{2\pi}\log|x| &\text{si $n = 2$}\\
\frac{1}{n(n-2)\omega_n|x|^{n-2}}&\text{si $n \geq 3$}
\end{cases}
\]
que \guill{concentra toda la masa en el cero}. De hecho, en el sentido distribucional $\Phi$ es una solución fundamental para el operador $-\Delta$. Sabemos también que si $f \in \C_c^2(\R^n)$, entonces la función $u := \Phi \ast f$ es de clase $\C^2(\R^n)$ y satisface la ecuación de Poisson,
\[
-\Delta u = f.
\]

El objetivo de esta presentación es probar el teorema de Malgrange-Ehrenpreis, que garantiza la existencia de soluciones fundamentales para cualquier operador diferencial parcial lineal con coeficientes constantes no nulo. A partir de este resultado y extendiendo la noción de convolución al contexto de las distribuciones, como en el caso del laplaciano se puede concluir que toda ecuación de la forma
\[
\L u = f \tag{$1$}
\]
tiene una solución en $\dist$. De hecho, si $f$ es suave la solución obtenida también lo es. 
\subsection{Sobre la Notación}

Si $q = \sum_{\alpha}c_\alpha x^\alpha\in \R[X_1, \dots, X_n]$ es un polinomio de grado $m$, escribiremos
\[
|q|_k = \max\{|c_\alpha| : |\alpha| = k\}
\]
para cada $k \leq m$.

Dado $U \subset \R^n$, notaremos $\ip{,}$ y $\|\cdot\|$ al producto interno y norma de $L^2(U)$ respectivamente. Indicaremos el dominio sólo cuando no se deduzca del contexto. 

La expresión \guill{operador diferencial} y el símbolo $\L$ referirán siempre a un operador diferencial parcial lineal (no nulo) con coeficientes constantes, y escribiremos $q(D)$ al operador asociado a un polinomio $q \in \R[X_1, \dots, X_n]$. 

\section{\scshape La desigualdad de Hörmander y soluciones en $L^2$}

El resultado de principal utilidad para relacionar las normas en $L^2$ de una función test y su imagen por un operador diferencial será el siguiente,

\begin{theorem}[Hörmander] Sae $\Omega \subset \R^n$ un abierto acotado y $\L = p(D)$ un operador diferencial. Existe entonces $C > 0$ tal que
\[
\|\L\varphi\| \geq C\|\varphi\|
\]
para toda $\varphi \in \mathscr{C}_c^\infty(\Omega)$, y ésta depende sólamente de los términos de orden máximo de $p$ y el diámetro de $\Omega$.
\end{theorem}
\begin{proof} Notemos $\rho := \sup_{\Omega}\|x\|$. Para cada $j \in \nat{n}$, definimos $p_j(D)$ como el operador diferencial que satisface la ecuación
\[
p(D)(x_j\varphi) = x_jp(D)\varphi + p_j(D)\varphi
\]
para toda función $\varphi \in \C_c^\infty(\R^n)$. Como dado $\alpha$ un multiíndice y $\varphi \in \C_c^\infty(\R^n)$ es
\[
D^\alpha x_j\varphi = x_j \cdot \sum_{\beta \leq \alpha, \beta_1 = 0} {\alpha \choose \beta} D^{\alpha - \beta}\varphi + \sum_{\beta \leq \alpha, \beta_1 = 1} {\alpha \choose \beta}D^{\alpha - \beta}\varphi,
\]
el operador $p_j(D)$ está bien definido y viene inducido por un polinomio, que notaremos $p_j$. Además, de la ecuación anterior se observa que $p_j(D)$ es cero si y sólo si $p$ no tiene monomios que contengan a la variable $X_j$. Dicho de otra forma, el operador diferencial $p_j(D)$ es nulo si y sólo si \guill{en $p(D)$ no hay derivadas con respecto a la $j$-ésima variable}.

De ser no nulo, el orden de $p_j$ debe ser menor al de $p$, y existe $j \in \nat{n}$ tal que $p_j$ es de orden exactamente $m-1$ y $|p_j|_{m-1} \geq |p|_m$.\todo{¿por qué?} Esto nos dice que para terminar la demostración alcanzaría probar que para todo $j \in \nat{n}$ es
\[
\|p_j(D)\varphi\| \leq 2mA\|p(D)\varphi\| \tag{$\ast$}.
\]
En efecto, de valer $\tpaint{$(\ast)$}$ sería
\[
\|(((p_{j_1})_{j_2})\cdots)_{j_k}(D)\varphi\| \leq 2A \cdots 2(m-1)A2mA\|p(D)\varphi\| = m!(2A)^m \|p(D)\varphi\|
\]
con $q = (((p_{j_1})_{j_2})\cdots)_{j_k}$ de grado $0$, así que
\[
\|q\|_0 \geq \cdots \geq \|p\|_m
\]
y entonces
\[
\|p(D)\varphi\| \geq \frac{1}{m!(2A)^m}\|q\varphi\| = \frac{1}{m!(2A)^m}\|q\|_0 \|\varphi\| \geq \frac{\|p\|_m}{m!(2A)^m}\|\varphi\|. 
\]


Para terminar, fijemos $j \in \nat{n}$ y veamos $\tpaint{($\ast$)}$ por inducción en $m$. Si $m = 0$, entonces $P_j(D)$ es nulo y la desigualdad se satisface. Supongamos ahora que $m > 1$ y el enunciado es cierto para polinomios de grado $m-1$. La desigualdad junto con la definición de $p_j(D)$ implican
\[
\|p(D)(x_i\varphi)\| \leq (2m+1)A\|p(D)\varphi\| \tag{$\lozenge$}
\]
para cualquier $i \in \nat{n}$. Aplicando esto a $p_j$, se obtiene entonces 
\[
\|p_j(D)\varphi\| \leq (2m-1)A\|p_j(D)\varphi\|.
\]
Ahora, observemos\footnote{Usamos aquí que los operadores diferenciales conmutan entre sí, y en particular es $\|q(D)\varphi\|^2 = \|q(D)^\ast \varphi\|^2$.} que
\[
\ip{p(D)(x_j\varphi),p_j(D)\varphi} = \ip{x_jp(D)\varphi,p_j(D)\varphi} + \|p_j(D)\varphi\|^2,
\]
y entonces
\begin{align*}
\|p_j(D)\varphi\|^2 &= \ip{p(D)(x_j\varphi),p_j(D)\varphi} - \ip{x_jp(D)\varphi,p_j(D)\varphi}\\
&= \ip{p_j(D)^\ast(x_j\varphi),p(D)^\ast\varphi} - \ip{x_jp(D)\varphi,p_j(D)\varphi}.
\end{align*}
Aplicando $\tpaint{($\lozenge$)}$ a $p_j(D)^\ast$ es
\begin{align*}
\|p_j(D)\varphi\|^2&\leq (2m-1)A\|p_j(D)^\ast\varphi\|\|p(D)^\ast\varphi\| + \|x_jp(D)\varphi\|\|p_j(D)\varphi\|\\
& \leq (2m-1)A\|p_j(D)\varphi\|\|p(D)\varphi\| + A\|p(D)\varphi\|\|p_j(D)\varphi\|\\
& = 2mA\|p_j(D)\varphi\|\|p(D)\varphi\|,
\end{align*}
lo que concluye la demostración.
\end{proof}

A partir de esta desigualdad, se obtiene inmediatamente la existencia de soluciones de $(1)$ en $L^2$ para abiertos acotados,

\begin{corollary} Sea $\Omega \subset \R^n$ un abierto acotado. Dada $g \in L^2(\Omega)$ y $\L = p(D)$ un operador diferencial, existe una solución $u \in L^2(\Omega)$ a la ecuación
\[
\L u = g.
\]
\end{corollary}
\begin{proof} Por la desigualdad de Hörmander, la (co)restricción de un operador diferencial a $\C_c^\infty(\Omega)$ resulta una función lineal inyectiva con inversa continua respecto de la norma en $L^2(\Omega)$. Notando $E := \im \L^\ast$, esto significa que el operador
\[
(\L^\ast)^{-1} : \L^\ast \varphi \in E \mapsto \varphi \in L^2(\Omega)
\]
está bien definido y resulta continua con respecto a la norma (en el caso de $E$, inducida) de $L^2(\Omega)$. Componiendo con el funcional representado por $g$, se obtiene un funcional
\[
\eta  : \L^\ast\varphi \in E \mapsto \ip{\varphi, g} \in \R,
\]
que tiene una única extensión $\widetilde{\eta}$ a $\overline{E}$, pues $\overline{E}$ es de Hilbert al ser un subespacio cerrado de $L^2(\Omega)$. Por el teorema de representación de Riesz, existe entonces $u \in \overline{E} \subset L^2(\Omega)$ tal que
\[
\ev{g,\varphi} = (\varphi,g) = (u,\L^\ast\varphi) = \ev{u,\L^\ast\varphi} = \ev{\L u, \varphi}
\]
para toda $\varphi \in \C_c^\infty(\Omega)$, y esto prueba finalmente que $\L u = g$.
\end{proof}

\section{\scshape La relación entre $\sop u$ y $\sop \L u$}

\begin{remark} Si $\L = p(D)$ es un operador diferencial y $\eps > 0$, entonces
\[
\widetilde{\L}_\eps(\psi) = e^{\eps x_1}\L(e^{-\eps x_1}\psi)
\]
es un operador diferencial. Basta verlo para $\L = D^\alpha$, pues la asignación $\L \mapsto \widetilde{\L}_\eps$ es lineal. Procedemos por inducción en $\alpha_1$, donde el caso base es cierto pues $\widetilde{D^{(0,\alpha')}} = D^{(0,\alpha')}$ . Si $\alpha_1 > 0$ y $\widetilde{D^{(\alpha_1 - 1, \alpha')}} = q(D)$
\begin{align*}
e^{\eps x_1}D^\alpha(e^{-\eps x_1}\psi) &= e^{\eps x_1}D^{(\alpha_1 - 1, \alpha')}\left(\frac{\partial}{\partial x_1}(e^{-\eps x_1}\psi)\right)\\
&= e^{\eps x_1}D^{(\alpha_1 - 1, \alpha')}\left(-\eps e^{-\eps x_1}\psi + e^{-\eps x_1}\frac{\partial}{\partial x_1}\psi\right)\\
&= e^{\eps x_1}D^{(\alpha_1 - 1, \alpha')}(e^{-\eps x_1}(-\eps\psi)) + e^{\eps x_1}D^{(\alpha_1 - 1, \alpha')}(e^{-\eps x_1}\partial_1\psi)\\
&= -\eps q(D)(\psi) + q(D)(\partial_1\psi) = -\eps q(D)(\psi) + (X_1q)(D)(\psi).
\end{align*}
Más todavía, los términos de orden mayor de $\widetilde{\L}_\eps$ coinciden con los de $\L$ pues
\[
e^{\eps x_1}D^\alpha (e^{-\eps x_1}\varphi) = e^{\eps x_1}\sum_{\beta \leq \alpha}{\alpha \choose \beta}D^\alpha e^{-\eps x_1}D^{\alpha - \beta}\varphi,
\]
y el término de orden máximo de esta expresión es $e^{\eps x_1}D^{\mathbf{0}}e^{-\eps x_1}D^{\alpha}\varphi = D^\alpha\varphi$. De este modo \guill{podemos tomar la misma constante que para $\L$ en la desigualdad de Hörmander}.
\end{remark}

\begin{proposition} \label{sharp-hormander-bound}Sea $\Omega \subset \R^n$ un abierto acotado y $\L = p(D)$ un operador diferencial. Existe $C' > 0$ tal que para todo $\eta \in \R$ es
\[
\int_{\Omega}e^{\eta x_1}|\L \varphi|^2 \geq C \int_{\Omega}e^{\eta x_1}|\varphi|^2
\]
para toda $\varphi \in \C_c^\infty(\Omega)$.
\end{proposition}
\begin{proof} Sea $C > 0$ una constante que satisfaga la desigualdad de Hörmander para $\L$. Por la observación anterior, podemos tomar $C$ de forma que para cada $\eta \in \R$ poniendo $\eps = \eta/2$ es
\[
\|\widetilde{\L}_\eps( e^{\eps x_1}\varphi)\| \geq C \|e^{\eps x_1}\varphi\|.
\]
Elevando al cuadrado esta desiguadad, se obtiene precisamente que
\[
\int_{\Omega}e^{\eta x_1}|\L \varphi|^2 \geq C \int_{\Omega}e^{\eta x_1}|\varphi|^2.
\]
\end{proof}

\begin{corollary} Si $\L = p(D)$ es un operador diferencial y $\varphi \in C_c^\infty(\R^n)$ es tal que $\L \varphi$ se anula en un semiespacio abierto $H$, entonces $\varphi$ se anula en $H$.
\end{corollary}
\begin{proof} Veámoslo primero para $H_0 := \{x_1 > 0\}$. Basta ver que $\|\varphi\|_{H_0} = \|\varphi\|_{\Omega \cap H_0} = 0$. En efecto, si $\Omega \supset \sop \varphi$, por la Proposición \ref{sharp-hormander-bound} es
\begin{align*}
0 &\leq C\int_{H_0 \cap \Omega}|\varphi|^2 \leq C\int_{H_0\cap \Omega}e^{\eta x_1}|\varphi|^2 \leq \int_{\Omega}e^{\eta x_1}|\L \varphi|^2 - C\int_{H_0^c \cap \Omega}e^{\eta x_1}|\varphi|^2\\
&= \int_{H_0^c \cap \Omega}e^{\eta x_1}|\L \varphi|^2 - C\int_{H_0^c \cap \Omega}e^{\eta x_1}|\varphi|^2, 
\end{align*}
y el lado derecho tiende a cero cuando $\eta \to \infty$. 

\todo{Revisar!}Si ahora $H$ es un hiperplano cualquiera, rotando y trasladando obtenemos un difeomorfismo suave $\psi : \R^n \to \R^n$ que envía $H_0$ a $H$. Como $\sop \L (\varphi \circ \psi) \subset \sop (\L \varphi) \circ \psi$, sabemos que $\L \varphi \circ \psi$ se anula en $H_0$, y por lo tanto así lo hace $\varphi \circ \psi$. Esto concluye la demostración.
\end{proof}

\begin{corollary} Sea $\L = p(D)$ un operador diferencial y $\varphi \in C_c^\infty(\R^n)$. Si $\L \varphi$ está soportada en $B_r(0)$, entonces $\varphi$ está soportada en $B_r(0)$.
\end{corollary}
\begin{proof} Podemos escribir a $B_r(0)$ como intersección de semiespacios. En el complemento de cada uno, sabemos que $\L \varphi$ se anula, así que $\varphi$ también lo hace.
\end{proof}

\begin{proposition}Sea $\L = p(D)$ un operador diferencial y $f \in L^2(\R^n)$ de soporte compacto. Si $\L f$ está soportada en $B_r(0)$, entonces $f$ está soportada en $B_r(0)$.
\end{proposition}
\begin{proof} Basta ver que existen $(f_\eps)_{\eps > 0} \subset L^2(\R^n)$ tales que $\sop f_\eps \subset B_{r+\eps}(0)$ y $f_\eps \xrightarrow{L^2} f$. En tal caso, si $\varphi \in \C_c^\infty(\R^n)$ está soportada en un compacto $K \subset B_r(0)^c$, existe $\mu > 0$ tal que $K \subset B_{r+\mu}(0)^c$ y por lo tanto
\[
(f,\varphi) = \lim_{\eps \to 0}(f_\eps,\varphi) = 0
\]
ya que $\sop f_\eps \cap K = \varnothing$ para $\eps \ll \mu$.

Dado $\chi$ un núcleo regularizante, definimos $f_\eps = f \ast \chi_\eps$. Esto garantiza la convergencia: veamos para terminar que $f_\eps$ está soportada en $B_{r+\eps}(0)$. Como éstas son ahora una funciones suaves, basta ver que $\L f_\eps$ está soportada en $B_{r+\eps}(0)$ para cada $\eps >0$. Por definición es
\[
\L f \ast \chi_\eps(z) = \ev{\L f,\chi_\eps(z-\cdot)} = \ev{f,\L^\ast\chi_\eps(z-\cdot)} = \ev{f,(\L\chi_\eps)(z-\cdot)} = f \ast (\L \chi_\eps
)(z) = \L f_\eps(z),
\]
así que si $\L f_\eps(z) \neq 0$ necesariamente $\sop \chi_\eps(z - \cdot) \subset B_\eps(z)$ no puede estar contenido fuera de $B_r(0)$, y esto implica que $z \in B_{r+\eps}(0)$.

\end{proof}

\section{\scshape Aproximación y soluciones en $L^2_{loc}$}

\begin{definition} Sean $0 < r < R$ y $\L = p(D)$ un operador diferencial. Notaremos
\[
N_{r,R}^\L = \{v|_{B_r(0)} : v \in L^2(B_R(0)), \ \ \L v = 0\}
\]
al subespacio de $L^2(B_r(0))$ que consiste de restringir funciones que anulan a $\L$ en $L^2(B_R(0))$.
\end{definition}

\begin{lemma} \label{lema-cota-lp-r-R}Sean $0 < r < R$ y $\L = p(D)$ un operador diferencial. Si $g \in (N_{r,R,\L})^\perp$, existe $C > 0$ tal que
\[
|(\varphi,g)_{B_r(0)}| \leq C\|\L \varphi\|_{B_R(0)}
\]
para toda $\varphi \in \C_c^\infty(\R^n)$.
\end{lemma}
\begin{proof} Si $\L \varphi = 0$ en $B_R(0)$, sabemos que $\varphi = 0$ allí y por lo tanto $(\varphi,g)_{B_r(0)} = 0$. Si no, existe $\psi \in L^2(B_R(0))$ tal que $\L \psi = \L \varphi$. Además, existe \todo{esto es lo que falta ver!!!} $C' > 0$ tal que $\|\psi\|_{B_R(0)} \leq C'\|\L \varphi\|_{B_R(0)}$ independiente de $\psi$ y $\varphi$ \todo{Y esto, si no, no anda}. Como $g \in (N_{r,R,\L})^\perp$, es
\[
\ip{g,\varphi}_{B_r(0)} = \ip{g,\varphi-\psi}_{B_r(0)} + \ip{g,\psi}_{B_r(0)} = \ip{g,\psi}_{B_r(0)}
\]
y por lo tanto
\[
|(\varphi,g)_{B_r(0)}| = |\ip{g,\psi}_{B_r(0)}| \leq \|g\| \cdot \|\psi\|_{B_r(0)} \leq \|g\|C' \cdot \|\L \varphi\|_{B_R(0)},
\]
así que poniendo $C = \|g\|C'$ se obtiene la desigualdad.
\end{proof}

\begin{lemma} \label{lema-rep} Sean $0 < r < R$ y $\L = p(D)$ un operador diferencial. Dada $g \in (N_{r,R,\L})^\perp$, existe $w\in L^2(B_R(0))$ tal que
\[
(\varphi,g)_{B_r(0)} = (\L \varphi, w)_{B_R(0)}
\]
para toda $\varphi \in \C_c^\infty(\R^n)$.
\end{lemma}
\begin{proof} Consideremos $E = \{ \L\psi|_{B_R(0)} : \psi \in \C_c^\infty(\R^n)\}$ como subespacio de $L^2(B_R(0))$. El lema anterior nos dice que la aplicación 
\[
\eta : \L \varphi \in E \mapsto \ip{\varphi,g}_{B_r(0)} \in \R
\]
está bien definida, y más aún define un funcional continuo. En particular podemos extenderlo a un funcional $\widetilde{\eta} : \overline{E} \subset L^2(B_R(0)) \to \R$ y como éste es de Hilbert, por el teorema de representación de Riesz existe $w \in \overline{E} \subset L^2(B_R(0))$ tal que $\widetilde{\eta} \equiv (w,-)_{B_R(0)}$. Esto termina de probar que
\[
(\varphi,g)_{B_r(0)} = \eta(\L \varphi) = (w,\L \varphi)_{B_R(0)}
\]
para toda $\varphi \in \C_c^\infty(\R^n)$.
\end{proof}

\begin{proposition} \label{aprox} Sean $0 < r < r' < R$. Dado un operador diferencial $\L = p(D)$ y $v \in L^2(B_{r'}(0))$ tal que $\L v = 0$ en $B_{r'}(0)$, existe una sucesión $(v_j)_{j \geq 1} \subset L^2(B_R(0))$ que satisface $\L v_j = 0$ en $B_R(0)$ y 
\[
v_j \xrightarrow{L^2(B_r(0))} v.
\]
\end{proposition}
\begin{proof} Supongamos primero que $v$ es suave y de soporte compacto. Notando
\[
S = \{ w|_{B_r(0)} : w \in L^2(B_R(0)), \ \L w = 0 \text{ en } B_R(0)\} \leq L^2(B_r(0)),
\]
basta ver que $S$ es denso en $T := \langle S, v|_{B_r(0)} \rangle$. A su vez, para ello alcanza probar que un funcional $f : T \to \R$ que se anula en $S$ es nulo. Un tal funcional se extiende a $\overline{T}$ y por el teorema de representación de Riesz, existe cierta $g \in L^2(B_r(0))$ tal que $f \equiv (g,-)$. 

En definitiva, es suficiente probar que si $g \in L^2(B_r(0))$ es tal que 
\[
\ip{g,w}_{B_r(0)} = 0
\]
para toda $w \in S$, entonces $\ip{g,v}_{B_r(0)} = 0$. Por el Lema $\tpaint{\ref{lema-rep}}$, sabemos que existe  $w \in L^2(B_R(0))$ tal que $(\varphi,g)_{B_r(0)} = (\L \varphi, w)_{B_R(0)}$ para toda $\varphi \in \C_c^\infty(\R^n)$. Considerando las extensiones por cero $\widetilde{g},\widetilde{w} \in L^2(\R^n)$ de $g$ y $w$ respectivamente, por definición es
\[
\ev{\L^\ast \widetilde{w},\varphi} = (\widetilde{w},\L \varphi) = (w,\L \varphi)_{B_R(0)} = (g,\varphi)_{B_r(0)} = \ev{ g, \varphi}
\]
y por lo tanto $\L^\ast \widetilde{w} = \widetilde{g}$. Como $\widetilde{g}$ tiene soporte en $B_r(0)$ y $\widetilde{w}$ tiene soporte compacto, luego $\widetilde{w}$ está soportada en $B_r(0)$. En particular $w$ está soportada en $B_r(0)$, y como $\L v$ se anula allí, es 
\[
(v,g)_{B_r(0)} = (\L v, w)_{B_R(0)} = (\L v, w)_{B_r(0)} = 0.
\]

Para terminar, veamos que el resultado sigue siendo cierto cuando $v$ no es suave de soporte compacto. Convolucionando con $\chi_\eps$ donde $\chi$ es una aproximación de la identidad, conseguimos $v_\eps \to v$ en $L^2(B_R(0))$. Eventualmente para $\eps \ll 1$ podemos achicar\footnote{Como $\sop \L v_\eps = \sop \L v \ast \chi_\eps = \sop \L v + B_\eps(0)$ y $B_{r'}(0) \subset (\sop \L v)^c$, entonces $B_{r'-\eps}(0) \subset (\sop \L v_\eps)^ c$.} $r'$ de tal modo que $\L v_\eps$ se anule en una bola de radio $\eta_\eps \in (r,r']$, y aplicar el resultado para funciones suaves, consiguiendo así sucesiones $v_{j}^i \to v_{1/i}$ en $L^2(B_r(0))$ tales que $\L v_j^i = 0$ en $B_R(0)$ para cada $i \geq 1$. Ahora, para cada $i \geq 1$ existe $j_i \in \N$ tal que $\|v_{j_i}^i - v_{1/i}\| < 1/i$ y entonces
\[
\|v - v_{j_i}^i\| \leq \|v-v_{i/1}\| + \|v_{j_i}^i - v_{1/i}\| \leq \|v-v_{i/1}\| + 1/i \to 0.
\]
En consecuencia, la sucesión $(v_{j^i}^i)_{i \geq 1}$ satisface la condición buscada.
\end{proof}

\begin{theorem} \label{teo-sol-l2-loc}Sea $\L = p(D)$ un operador diferencial. Dada $g \in L^2_{loc}(\R^n)$, existe $u \in L^2_{loc}(\R^n)$ tal que $\L u = g$.
\end{theorem}
\begin{proof} Sea $u_1 \in L^2(B_2(0))$ una solución de $\L u = g$ allí. Inductivamente construiremos $u_{k+1}$ del siguiente modo: tomamos una solución $w$ de $\L u = g$ en $L^2(B_{k+2}(0))$, de forma que $\L (u_k-w) = 0$ en $B_{k+1}(0)$. Por la Proposición $\tpaint{\ref{aprox}}$, existe $v \in L^2(B_{k+2}(0))$ tal que $\L v = 0$ y $\|u_k-w-v\|_{B_{k}(0)} < 1/2^k$. Definiendo $u_{k+1} := v+w$, la sucesión $(u_k)_{k \geq 1}$ satisface
\begin{itemize}
\item $\L u_k = g$ en $B_{k+1}(0)$, y
\item $\|u_{k+1} - u_k\|_{B_k(0)} < 1/2^k$.
\end{itemize} 

Veamos que $(u_k)_{k \geq 1}$ es de Cauchy en $L_{loc}^2(\R^n)$. Fijemos $K \subset \R^n$ compacto. Existe entonces $k_0 \in \N$ tal que $B_k(0) \supset K$ si $k > k_0$. Por lo tanto, si $m > n > k$ es
\begin{align*}
\|u_m-u_n\|_K &\leq \|u_m-u_{m-1}\|_K + \dots + \|u_{n+1}-u_n\|_K\\
&\leq \|u_m-u_{m-1}\|_{B_m(0)} + \dots + \|u_{n+1}-u_n\|_{B_n(0)}\\
&\leq 1/2^{m-1} + \dots + 1/2^n \leq \sum_{j \geq n}1/2^j,
\end{align*}
y esto tiende a cero si $n \to \infty$. Para terminar, veamos que $u := \lim_{k \to \infty} u_k$ satisface $\L u = g$. En efecto, si $\varphi \in \C_c^\infty(\R^n)$ entonces
\begin{align*}
\ev{\L u, \varphi} &= (u,\L^\ast\varphi)_{\sop \varphi} = \lim_{k \to \infty}(u_k,\L^\ast\varphi)_{\sop \varphi} = \lim_{k \to \infty}(u_k,\L^\ast\varphi)_{B_{k+1}(0)}\\
&= \lim_{k \to \infty}(g,\L^\ast\varphi)_{B_{k+1}(0)} = (g,\L^\ast\varphi)\\
&= \ev{\L g, \varphi}.
\end{align*}
\end{proof}

\section{\scshape Existencia de soluciones fundamentales y consecuencias}

\begin{lemma} La función $H : \R^n \to \R$ definida por $H(x_1,\dots, x_n) = \prod_{i=1}^n \chi_{\{x_i > 0\}}$ es un elemento de $L_{loc}^2(\R^n)$ y satisface
\[
\frac{\partial^n}{\partial x_1 \cdots \partial x_n} H = \delta_0.
\]
\end{lemma}
\begin{proof} Si $\varphi \in \test$, es
\begin{align*}
\int_0^\infty \frac{\partial}{\partial x_i}\varphi(x_1,\dots, x_i,\dots,x_n)dx_i &= \lim_{t \to \infty} \varphi(x_1,\dots,\overbrace{t}^i,\dots,x_n) - \varphi(x_1, \dots,0,\dots,x_n)\\
& = - \varphi(x_1, \dots,0,\dots,x_n),
\end{align*}
ya que $\varphi$ tiene soporte compacto. Por lo tanto, es
\begin{align*}
\left\ev{\frac{\partial^n}{\partial x_1 \cdots \partial x_n} H,\varphi\right} &= \left\ev{H,(-1)^n \cdot \frac{\partial^n}{\partial x_1 \cdots \partial x_n}\varphi\right} = (-1)^n\int_{\R^n}\frac{\partial^n}{\partial x_1 \cdots \partial x_n}\varphi \cdot dx_1 \cdots dx_n\\
&= (-1)^n\int_0^\infty \cdots \ \ \int_0^\infty\frac{\partial^n}{\partial x_1 \cdots \partial x_n}\varphi \cdot dx_1 \cdots dx_n\\ &= (-1)^n(-1)^n\varphi(0,\dots,0) = \varphi(0) = \ev{\delta_0,\varphi}.
\end{align*}
\end{proof}

\begin{theorem}[Malgrange-Ehrenpreis] Todo operador diferencial parcial lineal no nulo con coeficientes constantes admite una solución fundamental.
\end{theorem}
\begin{proof} Sea $\L = p(D)$ un operador diferencial. Por el $\tpaint{Teorema \ref{teo-sol-l2-loc}}$, existe $u \in L^2_{loc}(\R^n)$ tal que $\L u = H$. Considerando la distribución $\frac{\partial^n}{\partial x_1 \cdots \partial x_n} u$ se tiene efectivamente que
\[
\L \left(\frac{\partial^n}{\partial x_1 \cdots \partial x_n} u\right) = \frac{\partial^n}{\partial x_1 \cdots \partial x_n} \L u = \frac{\partial^n}{\partial x_1 \cdots \partial x_n} H = \delta_0.
\]
\end{proof}

\begin{lemma} Si $f \in \test$, entonces $\delta_0 \ast f = f$.
\end{lemma}
\begin{proof} Por definición, es
\[
\delta_0 \ast f(x) = \ev{\delta_0,f(x -\cdot)} = \int_{\R^n} f(x-\cdot) d \delta_0 = f(x - 0) = f(x).
\]
\end{proof}

\begin{corollary} Sea $f \in \test$ y $\L = p(D)$ un operador diferencial. Entonces la ecuación $\L u = f$ tiene una solución en $\test$.
\end{corollary}
\begin{proof} Sea $s$ una solución fundamental para $\L$. Si ponemos $u := s \ast f$, esto define una función test y
\[
\L (s \ast f) = (\L s) \ast f = \delta_0 \ast f = f,
\]
lo que concluye la demostración.
\end{proof}

\begin{example} Notemos que el corolario anterior describe además como encontrar soluciones suaves a partir de una solución fundamental. Usemos esto para tratar un caso ya conocido: hallemos soluciones suaves a la ecuación
\[
b_1 \cdot \frac{\partial u}{\partial x_1} + \cdots + b_n \cdot \frac{\partial u}{\partial x_n} = \beta u + g \tag{2}
\]
con $g \in \C_c^\infty(\R^n), b = (b_1, \dots, b_n) \in \R$ y $\beta \in \R_{ < 0}$. 

Observemos que notando $p = \sum_{i=1}^n b_i X_i - \beta$ y $\L = p(D)$, la ecuación se puede rescribir como
\[
\L u = g.
\]
Busquemos en primer lugar la solución fundamental del operador $\L$. Proponemos 
\[
\ev{u,\varphi} := \int_{0}^{+\infty}\varphi(tb)e^{\beta t}dt,
\]
ya que
\begin{align*}
\ev{\L u, \varphi} &= \ev{u,\L^\ast \varphi} = - \sum_{i=1}^n b_i \int_{0}^{+\infty}\varphi_i(tb)e^{\beta t}dt + \beta \int_0^{\infty}\varphi(tb)e^{\beta t}dt\\
&= -\int_{0}^{+\infty}\left(\sum_{i=1}^n b_i \varphi_i(tb)\right)e^{\beta t}dt +  \int_0^{\infty}\varphi(tb)\beta e^{\beta t}dt\\
&= - \int_{0}^{+\infty} \frac{d}{dt}\varphi(tb) e^{\beta t} + \varphi(tb)\frac{d}{dt}e^{\beta t}dt\\
&= -\int_{0}^{+\infty}\frac{d}{dt}(\varphi(tb)e^{\beta t})dt = \varphi(tb)e^{\beta t} \Bigg|_{+\infty}^0 = \varphi(0).
\end{align*}
para cada $\varphi \in \C_c^\infty(\R^n)$. Por lo tanto, una solución para $\tpaint{(2)}$ es
\[
v(x) = u \ast g(x) = \ev{u, g(x - \cdot)} = \int_{0}^{+\infty}g(x-tb)e^{\beta t}dt.
\]
\qed
\end{example}

\subsection{Apéndice: sobre la convolución de distribuciones}

\begin{definition} Sean $T \in \dist$ y $\varphi \in \test$. Se define la \textit{convolución} de $T$ y $\varphi$ como
\[
(T \ast \varphi)(x) := \ev{T,\varphi(x - \cdot)}.
\]
\end{definition}

\begin{definition} Sea $T \in \dist(U)$ y $V \subset U$ un abierto. Decimos que $T$ se anula en $V$ si para toda $\varphi \in \test$ con $\sop \varphi \subset V$ se tiene que $\ev{T,\varphi} = 0$.
\end{definition}

\begin{definition} El \textit{soporte} de una distribución $T \in \dist(U)$ es el abierto maximal\footnote{Esto está bien definido, y más aún se puede verificar que el soporte es el complemento de la unión de los abiertos donde $T$ se anula.} en $V$ tal que $T$ se anula en $V^c$. 
\end{definition}

\begin{theorem}[\cite{H}, Theorem 4.1.1]  Si $T \in \dist$ y $\varphi \in \test$, entonces $T \ast \varphi$ es una función suave de soporte compacto con $\sop T \ast \varphi \subset \sop T + \sop \varphi$ y
\[
D^\alpha (T \ast \varphi ) = (D^\alpha T) \ast \varphi = T \ast (D^\alpha \varphi)
\]
para todo multiíndice $\alpha$. \qed
\end{theorem}

\scshape
\begin{thebibliography}{}
\normalfont
\bibitem{R}{Jean-Pierre Rosay. \textit{A Very Elementary Proof of the Malgrange-Ehrenpreis Theorem}. The American Mathematical Monthly, 1991.}
\bibitem{H}{Lars Hörmander. \textit{The Analysis of Linear Partial Differential Operators I. }Springer, 1990.}
\end{thebibliography}
\end{document}